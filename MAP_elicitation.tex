\begin{filecontents}{preliminary.sty}
\ProvidesPackage{preliminary}
%\DeclareOption{draft}{%
  \AtBeginDocument{%
    \renewcommand\maketitlehookc
\ProcessOptions
\RequirePackage{titling}
\endinput
\end{filecontents}

\documentclass[12pt, a4paper]{article}
\usepackage{setspace}
\usepackage{ragged2e}
\usepackage[centertags,reqno]{amsmath}
\usepackage{amssymb}
\usepackage{graphics,subfigure}
\usepackage[dvips]{graphicx}
\usepackage[dvipsnames]{xcolor}
\usepackage[hidelinks]{hyperref}
\usepackage{appendix}
\usepackage{natbib}
\usepackage{verbatim,color}
\usepackage{pdflscape}
\usepackage[showframe=false]{geometry}
\usepackage{changepage}
\usepackage{xcolor}
\usepackage{eurosym}
\usepackage{textcomp}
\usepackage[open,openlevel=1]{bookmark}
\usepackage{multirow}
\usepackage{caption}
\usepackage{hyphenat}
\usepackage{listings} % To include a Stata .do file
\usepackage{pdfpages} % To include the instructions from a pdf file
\newcommand{\mybox}[2]{{\color{#1}\fbox{\normalcolor#2}}}
\doublespacing

% Exception to hyphenation
\hyphenation{par-ti-ci-pants}
\hyphenation{par-ti-ci-pant}
\hyphenation{Hy-po-the-sis}
\hyphenation{ex-pe-ri-ment}
\hyphenation{ex-pe-ri-ments}

% Allows bigger tables to be scaled down
\usepackage{adjustbox}

%From my paper with Raymond
\usepackage{tabularx,calc}
\usepackage{dcolumn}                    % Aligns tables on the decimal point
\newcolumntype{d}[1]{D{.}{.}{#1}}       %       Aligns on dot
\newcolumntype{.}{D{.}{.}{3.5}}         %       Somehow it works better
\newcolumntype{C}{@{\extracolsep{.6cm}}c@{\extracolsep{0pt}}}
\usepackage{threeparttable}
\usepackage{siunitx,booktabs}
\sisetup{
    detect-all,
    round-integer-to-decimal = true,
    group-digits             = true,
    group-minimum-digits     = 4,
    group-separator          = {\,},
    table-align-text-pre     = false,
    table-align-text-post    = false,
    input-signs              = + -,
    input-symbols            = {*},
    input-open-uncertainty   = ,
    input-close-uncertainty  = ,
    retain-explicit-plus
}

% Commands to name appendices Appendix A, Appendix B, etc.
\makeatletter
%% The "\@seccntformat" command is an auxiliary command
%% (see pp. 26f. of 'The LaTeX Companion,' 2nd. ed.)
\def\@seccntformat#1{\@ifundefined{#1@cntformat}%
   {\csname the#1\endcsname\quad}  % default
   {\csname #1@cntformat\endcsname}% enable individual control
}
\let\oldappendix\appendix %% save current definition of \appendix
\renewcommand\appendix{%
    \oldappendix
    \newcommand{\section@cntformat}{\appendixname~\thesection\quad}
}

%Adds text specifying it is a preliminary version
%\usepackage{preliminary}

\title{Testing the Elicitation Procedure \\ of the Minimum Acceptable Probability}
\author{Maria Polipciuc\thanks{Maastricht University. Email: \url{m.polipciuc@maastrichtuniversity.nl}. We thank Elias Tsakas and participants in the BEELab proposal meeting for valuable comments.} \and Martin Strobel\footnotemark[1]}
\date{\today	\vspace{1cm}}
\titlepage


\begin{document}
\begin{titlepage}
\clearpage
\maketitle
\thispagestyle{empty}


\begin{abstract}

\end{abstract}
\end{titlepage}


\section{Introduction}\label{sec:intro}
Individuals have often been found to prefer exposure to a randomly generated risk than to an equiprobable risk generated by an opponent in a strategic situation.
This strategic risk premium has been dubbed \textit{betrayal aversion} \citep{Bohnet2004}.
Many papers find that betrayal aversion is an important determinant of trust CITE.

A recent paper has shown theoretically that the elicitation procedure of minimum acceptable probabilities (MAPs, from which betrayal aversion is identified) used in most papers leaves the door open to potential confounds such as ``ambiguity attitudes, complexity, different beliefs, and dynamic optimization'' \citep{Li2020a} if players are not rational expected utility maximizers.
Moreover, a couple of empirical papers which use more stringent identification procedures for betrayal aversion by controlling for beliefs do not find betrayal aversion \citep{Fetchenhauer2012,Polipciuc2020}, or find it to play a role for trusting only when beliefs are far more optimistic than is generally the case \citep{Engelmann2021}.

In this note, we use an online experiment to measure how much of what has been called betrayal aversion is due to distributional dependence, regardless of the source of risk being random or strategic.
To do this, we show participants complete distributions over probabilities of the good (bad) outcome of a lottery, and ask them for their cutoff probability of the favorable outcome for preferring the lottery to a safe payoff.
Should the numerical example in \cite[Appendix A][]{Li2020a} be applicable to our setting, we expect participants to more readily accept the lottery when it comes from a distribution with a higher expected value.
We find the opposite to be true: the higher the expected value of the probability of the favorable outcome, the higher the minimum acceptable probability required by participants to accept the lottery.

While this is at odds with our expectations, it ties in with some results from the empirical literature on distributional dependence of willingness to pay (WTP).
To make this clear, we first explain how betrayal aversion is related to WTP.
\textcolor{red}{I need something here...}
Betrayal aversion is identified as the difference between two minimum acceptable probabilities: the minimum required probability of trustworthiness in order to trust in a trust game, and the minimum required probability of a favorable outcome in an equiprobable control game, where risk is generated by a randomization device.
The MAPs are elicited through a variant of the Becker-Degroot-Marschak (BDM) \citep{Becker1964} mechanism, which is an often used mechanism for eliciting valuations.
In the Becker-DeGroot-Marschak (BDM) mechanism, a potential buyer states the maximum price for which she is willing to buy a good.
A price is drawn, and if it is lower than or equal to the price she stated, she buys the good.
If the price is higher, she keeps her endowment and does not buy the good.
There a couple of differences between the `standard' BDM and MAP elicitation: (1) the auctioned good is a lottery, (2) instead of giving a maximum price for which they prefer the good to a safe payment, participants are asked to state a minimum probability of the favorable outcome of the lottery for which they prefer the lottery to a safe payment and (3) the underlying distribution of the probability of the favorable outcome is not explicitly uniform, as in the case of the distribution of potential prices for the good.

Theoretical literature has pointed out that the BDM mechanism is not incentive compatible if players are not rational expected utility maximizers \citep{Karni1987,Horowitz2006}.
This is because individuals face uncertainty regarding the price of the good and additional uncertainty about whether they will buy the good or not.
If their utility function is influenced by these uncertainties, changing the price distribution of the good might influence their MAP.

Several empirical papers find this to be the case: generally, the higher the expected price of the good, the higher the WTP \citep[for a short review of this literature, see]{Tymula2016}.



\section{Procedures}\label{sec:proced}
We use three distributions, which are ordered in terms of the expected payoff over the entire distribution, as their name suggests: the Good, the Bad, and the Uniform.
Two of the three distributions were selected to emulate treatments in papers on betrayal aversion.
The first of these two (The Uniform) has equal chances of occurrence for each of the possible lotteries.
We assume that this is what participants expect to face in treatments with computer drawn lotteries, unless specified otherwise.
The second (The Bad) has an overall chance of a high payoff equal to the percentage of trustworthy respondents in papers on betrayal aversion.
The distribution in the Good treatment mirrors the one in the Bad treatment: it has the same variance, and minus the skewness of the Bad distribution.

\textcolor{red}{include table describing distributions}

To make the task easier to understand, we represent lotteries via wheels of fortune.
In each treatment, participants see the entire distribution of lotteries in that treatment, sorted in ascending order by the probability of the favorable outcome.
Figure \textcolor{red}{XZY} below shows the distribution for The Good treatment.
Dark blue sectors symbolize the high payoff, light blue sectors---the low payoff.
Participants have to state a \textit{minimum acceptable probability} (MAP): the lowest chance of a favorable outcome of the lottery such that they prefer the lottery to a sure payoff.
Since there is evidence that participants have an easier time expressing choice using integers than probabilities \citep{Quercia2016}, we ask them to answer a question which requires an integer as an input.
Specifically, they have to answer: ``Which wheels would you like to spin for your bonus?'' by inserting an integer between 0 and 15 in the blank space in the sentence: ``I prefer to spin wheels which have at least \textit{[blank]} dark blue sectors.''

The experiment was conducted online using Qualtrics.
Participants were UK residents registered on a platform for conducting academic studies (Prolific).
Since the elicitation of MAPs is rather complex \citep{Quercia2016,Polipciuc2020}, we opted for participants who had at least a bachelor's degree.
The study was pre-registered at the AEA RCT Registry (https://doi.org/10.1257/rct.7776-1.1).
275 of the 450 participants answered the eliminatory comprehension questions correctly and completed the experiment.
Those who completed the experiment (did not complete the experiment) spent a median time of 12.4 (5.9) minutes and earned 5 (1) UK pounds.\footnote{
The high median earnings of those who completed the experiment are due to a coding error.
Instead of all decisions in all three treatments being equally likely to be selected for payment, the error lead to only The Good and The Uniform treatments being selected, \textcolor{red}{with the probabilities indicated in parentheses}. CHECK
Participants were informed about the error, which increased the payoffs of all participants who had completed the experiment.
}




Several papers CITE find that when dealing with complex risks, participants in experiments require an extra premium compared to simple risk aversion.
This premium is positively correlated with ambiguity aversion. (ARE EFFECT size similar?)


\section{Hypotheses}\label{sec:hyp}
Let $p^*$ be the true probability of the high payoff, whose distribution varies between treatments.
According to the calculation in \textcolor{red}{Appendix...}, we expect the following ordering of MAPs:

\noindent \textbf{Hypothesis 1} \quad \textit{The MAP in the Good treatment (more mass on high values of $p^*$) is lower than the MAP in the Uniform treatment (a uniform distribution over $p^*$), which is lower than the MAP in the Bad treatment (more mass on low values of $p^*$).}

\begin{equation}
MAP_G < MAP_U < MAP_B
\end{equation}

In the pre-analysis plan, we specified that the alternative hypothesis ($MAP_B < MAP_U < MAP_G$) could be true instead if participants anchor on visual cues of the distributions, such as the mean.

\section{Data and results}\label{sec:results}
First, we present summary statistics for all decisions, by treatment and by decision order.
Next, we run non-parametric tests and ordinary least squares regressions to test the hypothesis.

Table \ref{tab:stats} presents the average MAP by treatment over all decisions and by decision order.
Already from this table we can see that the hypothesis is not supported by the data.


\begin{table}[htbp]
\centering \caption{Descriptive statistics: MAPs by treatment}\label{tab:stats}
\begin{threeparttable}
\begin{tabular}
   {@{}
	l
	*4{S[table-format=+1.5, table-space-text-pre={**}, table-space-text-post={-**}]}
	@{}
	}
\toprule
	&{All	decisions}&{First decision}&{Second	decision}&{Third	decision}\\
\cmidrule{2-5}
The Good	&	9.531&       9.571&       9.458&	9.553\\
	&	(2.503)&     (2.270)&     (2.500)&	(2.750)\\
The Uniform	&	8.844&       8.890&       8.368&	9.227\\
	&	(2.382)&     (2.392)&     (2.119)&	(2.539)\\
The Bad	&	8.615&       8.093&       9.124&	8.512\\
	&	(2.522)&     (2.597)&     (2.491)&	(2.387)\\
\midrule
N	&	{825}&       {275}&       {275}&	{275}\\
\bottomrule
\end{tabular}
\begin{tablenotes}
\item \textit{Notes:} The table shows averages per treatment.
Each participant made three decisions in randomized order.
Standard deviations in parentheses.
Possible answers were integers between 0 and 15.
\end{tablenotes}
\end{threeparttable}
\end{table}

A non-parametric Page's L test confirms this: there is strong evidence that the ordering is the opposite to the one hypothesized ($MAP_B < MAP_U < MAP_G$, $p$-value<0.001).\footnote{
Page's L test has the null hypothesis that all possible orderings are equally likely.
The alternative hypothesis is that a specified order is the increasing order of alternatives.
The Stata command is $pagetrend$.
}




\begin{table}[htbp]
\centering \caption{Linear regressions on Minimum Acceptable Frequencies}\label{tab:reg}
\begin{threeparttable}
\begin{tabular}
   {@{}
	l
	*4{S[table-format=+1.5, table-space-text-pre={**}, table-space-text-post={-**}]}
	@{}
	}
\toprule
\textbf{Dependent variable:}& \multicolumn{4}{c}{Minimum	acceptable frequency}\\
&       {(1)}   &       {(2)}   &	{(3)}   &       {(4)}   \\
\cmidrule(rl){2-5}
The Good            &       0.687***&       0.687***&	0.687***&       0.687***\\
&     (0.099)   &     (0.099)   &	(0.099)   &     (0.099)   \\
The Bad             &      -0.229***&      -0.229***&	-0.229***&      -0.229***\\
&     (0.070)   &     (0.070)   &	(0.070)   &     (0.070)   \\
Age                 &               &       0.005   &	0.006   &       0.006   \\
&               &     (0.013)   &	(0.013)   &     (0.014)   \\
Male                &               &      -0.047   &	0.030   &      -0.029   \\
&               &     (0.286)   &	(0.284)   &     (0.283)   \\
Risk aversion (0--10)&               &               &	-0.172** &      -0.152** \\
&               &               &	(0.074)   &     (0.074)   \\
\textit{Sequence} &&&& \\
\qquad UBG                 &               &               &	&       0.490   \\
&               &               &	&     (0.408)   \\
\qquad BGU                 &               &               &	&      -0.008   \\
&               &               &	&     (0.434)   \\
\qquad GBU                 &               &               &	&       1.173***\\
&               &               &	&     (0.412)   \\
\qquad BUG                 &               &               &	&      -0.150   \\
&               &               &	&     (0.434)   \\
\qquad UGB                 &               &               &	&       0.469   \\
&               &               &	&     (0.433)   \\
Constant            &       8.844***&       8.696***&	9.520***&       9.066***\\
&     (0.144)   &     (0.460)   &	(0.593)   &     (0.627)   \\
\midrule
N                   &       {825}   &       {825}   &	{825}   &       {825}   \\
\bottomrule
\end{tabular}
\begin{tablenotes}
\item \textit{Notes:} Standard errors clustered at the individual level in parentheses.
The baseline treatment is The Uniform.
The baseline sequence is GUB (Good--Uniform--Bad).
Risk attitudes are measured on a 0--10 scale, where 0 is very risk averse and 10 is very risk loving. \\
* p $<$ 0.10, ** p $<$ 0.05, *** p $<$ 0.01.
\end{tablenotes}
\end{threeparttable}
\end{table}



\clearpage
\pagebreak
\bibliographystyle{apalike}
\bibliography{Communities}

\end{document}