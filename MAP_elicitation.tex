\begin{filecontents}{preliminary.sty}
\ProvidesPackage{preliminary}
%\DeclareOption{draft}{%
  \AtBeginDocument{%
    \renewcommand\maketitlehookc
\ProcessOptions
\RequirePackage{titling}
\endinput
\end{filecontents}

\documentclass[12pt, a4paper]{article}
\usepackage{setspace}
\usepackage{ragged2e}
\usepackage[centertags,reqno]{amsmath}
\usepackage{amssymb}
\usepackage{graphics,subfigure}
\usepackage[dvips]{graphicx}
\usepackage[dvipsnames]{xcolor}
\usepackage[hidelinks]{hyperref}
\usepackage{appendix}
\usepackage{natbib}
\usepackage{verbatim,color}
\usepackage{pdflscape}
\usepackage[showframe=false]{geometry}
\usepackage{changepage}
\usepackage{xcolor}
\usepackage{eurosym}
\usepackage{textcomp}
\usepackage[open,openlevel=1]{bookmark}
\usepackage{multirow}
\usepackage{caption}
\usepackage{hyphenat}
\usepackage{listings} % To include a Stata .do file
\usepackage{pdfpages} % To include the instructions from a pdf file
\newcommand{\mybox}[2]{{\color{#1}\fbox{\normalcolor#2}}}
\doublespacing

% Exception to hyphenation
\hyphenation{par-ti-ci-pants}
\hyphenation{par-ti-ci-pant}
\hyphenation{Hy-po-the-sis}
\hyphenation{ex-pe-ri-ment}
\hyphenation{ex-pe-ri-ments}

% Allows bigger tables to be scaled down
\usepackage{adjustbox}

%From my paper with Raymond
\usepackage{tabularx,calc}
\usepackage{dcolumn}                    % Aligns tables on the decimal point
\newcolumntype{d}[1]{D{.}{.}{#1}}       %       Aligns on dot
\newcolumntype{.}{D{.}{.}{3.5}}         %       Somehow it works better
\newcolumntype{C}{@{\extracolsep{.6cm}}c@{\extracolsep{0pt}}}
\usepackage{threeparttable}
\usepackage{siunitx,booktabs}
\sisetup{
    detect-all,
    round-integer-to-decimal = true,
    group-digits             = true,
    group-minimum-digits     = 4,
    group-separator          = {\,},
    table-align-text-pre     = false,
    table-align-text-post    = false,
    input-signs              = + -,
    input-symbols            = {*} {**} {***},
    input-open-uncertainty   = ,
    input-close-uncertainty  = ,
    retain-explicit-plus
}

% Commands to name appendices Appendix A, Appendix B, etc.
\makeatletter
%% The "\@seccntformat" command is an auxiliary command
%% (see pp. 26f. of 'The LaTeX Companion,' 2nd. ed.)
\def\@seccntformat#1{\@ifundefined{#1@cntformat}%
   {\csname the#1\endcsname\quad}  % default
   {\csname #1@cntformat\endcsname}% enable individual control
}
\let\oldappendix\appendix %% save current definition of \appendix
\renewcommand\appendix{%
    \oldappendix
    \newcommand{\section@cntformat}{\appendixname~\thesection\quad}
}

%Adds text specifying it is a preliminary version
%\usepackage{preliminary}

\title{Testing the Elicitation Procedure \\ of the Minimum Acceptable Probability}
\author{Maria Polipciuc\thanks{Maastricht University. Email: \url{m.polipciuc@maastrichtuniversity.nl}. We thank Elias Tsakas and participants in the BEELab proposal meeting for valuable comments.} \and Martin Strobel\footnotemark[1]}
\date{\today	\vspace{1cm}}
\titlepage


\begin{document}
\begin{titlepage}
\clearpage
\maketitle
\thispagestyle{empty}


\begin{abstract}

\end{abstract}
\end{titlepage}


\section{Introduction}\label{sec:intro}
Imagine a lottery and a sure payment.
You know how much you receive if you win or lose the lottery, but you don't know the chances of each outcome.
Now imagine you have to write a contract specifying how good the lottery minimally has to be for you to prefer the lottery over the sure payment.
Would the requirement you set in the contract be independent of how likely you think more (less) favorable lotteries are?
As an example: would you set the same requirement in two situations, one in which favorable lotteries are more likely, and the other in which unfavorable lotteries are more likely?

From a theoretical point of view, things are clear.
If you are a rational expected utility maximizer, the requirement should be the same in both situations.
If you're not and the context influences your decisions, it is not so clear which way things will go.
Will you require a better or a lesser chance to take the lottery in a world of better opportunities?

This note reports the findings of an online experiment designed to test whether a more favorable underlying distribution of the chances of the lottery influences the threshold people require.
Our interest in this question was sparked by previous work on betrayal aversion.
Betrayal aversion has been identified as one of the factors influencing the decision to trust.
It is defined as an anticipatory disutility from expecting one's trust to be betrayed.
Betrayal aversion has been identified as the premium required to trust someone relative to accepting an equiprobable lottery with equal payoff consequences for an uninvolved other.

Most older papers on the subject of betrayal aversion find a positive strategic premium.
Some papers however do not.
\cite{Li2020} propose that the original BZ design might have miss-classified the premium as betrayal aversion.
They argue that, should participants not be rational expected utility maximizers, the premium could be attributed to ``ambiguity attitudes, complexity, different beliefs, and dynamic optimization''.

In this note, we examine the effect of one of these potential confounds: different underlying distributions of the lottery's winning chances.

\section{Procedures}\label{sec:proced}
We use three distributions, which are ordered in terms of the expected payoff over the entire distribution, as their name suggests: the Good, the Bad, and the Uniform.
Two of the three distributions were selected to emulate treatments in papers on betrayal aversion.
The Uniform treatment has equal chances of occurrence for each of the possible lotteries.
We assume that this is what participants expect to face in treatments with computer drawn lotteries, unless specified otherwise.
The Bad treatment has an overall chance of a high payoff equal to the percentage of trustworthy respondents in papers on betrayal aversion.
The distribution in the Good treatment mirrors the one in the Bad treatment: it has the same variance, and minus the skewness of the Bad distribution.

\textcolor{red}{include table describing distributions}

Participants have to state a \textit{minimum acceptable probability} (MAP): the lowest chance of a favorable outcome of the lottery such that they prefer the lottery to a sure payoff.

Several papers CITE find that when dealing with complex risks, participants in experiments require an extra premium compared to simple risk aversion.
This premium is positively correlated with ambiguity aversion. (ARE EFFECT size similar?)



\end{document}