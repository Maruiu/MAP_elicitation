\begin{filecontents}{preliminary.sty}
\ProvidesPackage{preliminary}
%\DeclareOption{draft}{%
  \AtBeginDocument{%
    \renewcommand\maketitlehookc
\ProcessOptions
\RequirePackage{titling}
\endinput
\end{filecontents}

\documentclass[12pt, a4paper]{article}
\usepackage{setspace}
\usepackage{ragged2e}
\usepackage[centertags,reqno]{amsmath}
\usepackage{amssymb}
\usepackage{graphics,subfigure}
\usepackage[dvips]{graphicx}
\usepackage[dvipsnames]{xcolor}
\usepackage[hidelinks]{hyperref}
\usepackage{appendix}
\usepackage{natbib}
\usepackage{verbatim,color}
\usepackage{pdflscape}
\usepackage[showframe=false]{geometry}
\usepackage{changepage}
\usepackage{xcolor}
\usepackage{eurosym}
\usepackage{textcomp}
\usepackage[open,openlevel=1]{bookmark}
\usepackage{multirow}
\usepackage{caption}
\usepackage{hyphenat}
\usepackage{listings} % To include a Stata .do file
\usepackage{pdfpages} % To include the instructions from a pdf file
\newcommand{\mybox}[2]{{\color{#1}\fbox{\normalcolor#2}}}
\doublespacing

% Exception to hyphenation
\hyphenation{par-ti-ci-pants}
\hyphenation{par-ti-ci-pant}
\hyphenation{Hy-po-the-sis}
\hyphenation{ex-pe-ri-ment}
\hyphenation{ex-pe-ri-ments}

% Allows bigger tables to be scaled down
\usepackage{adjustbox}

%From my paper with Raymond
\usepackage{tabularx,calc}
\usepackage{dcolumn}                    % Aligns tables on the decimal point
\newcolumntype{d}[1]{D{.}{.}{#1}}       %       Aligns on dot
\newcolumntype{.}{D{.}{.}{3.5}}         %       Somehow it works better
\newcolumntype{C}{@{\extracolsep{.6cm}}c@{\extracolsep{0pt}}}
\usepackage{threeparttable}
\usepackage{siunitx,booktabs}
\sisetup{
    detect-all,
    round-integer-to-decimal = true,
    group-digits             = true,
    group-minimum-digits     = 4,
    group-separator          = {\,},
    table-align-text-pre     = false,
    table-align-text-post    = false,
    input-signs              = + -,
    input-symbols            = {*} {**} {***},
    input-open-uncertainty   = ,
    input-close-uncertainty  = ,
    retain-explicit-plus
}

% Commands to name appendices Appendix A, Appendix B, etc.
\makeatletter
%% The "\@seccntformat" command is an auxiliary command
%% (see pp. 26f. of 'The LaTeX Companion,' 2nd. ed.)
\def\@seccntformat#1{\@ifundefined{#1@cntformat}%
   {\csname the#1\endcsname\quad}  % default
   {\csname #1@cntformat\endcsname}% enable individual control
}
\let\oldappendix\appendix %% save current definition of \appendix
\renewcommand\appendix{%
    \oldappendix
    \newcommand{\section@cntformat}{\appendixname~\thesection\quad}
}

%Adds text specifying it is a preliminary version
%\usepackage{preliminary}

\title{Testing the Elicitation Procedure \\ of the Minimum Acceptable Probability}
\author{Maria Polipciuc\thanks{Maastricht University. Email: \url{m.polipciuc@maastrichtuniversity.nl}. We thank Elias Tsakas and participants in the BEELab proposal meeting for valuable comments.} \and Martin Strobel\footnotemark[1]}
\date{\today	\vspace{1cm}}
\titlepage


\begin{document}
\begin{titlepage}
\clearpage
\maketitle
\thispagestyle{empty}


\begin{abstract}

\end{abstract}
\end{titlepage}


\section{Introduction}\label{sec:intro}
<<<<<<< HEAD

=======
Individuals have often been found to prefer exposure to a randomly generated risk than to an equiprobable risk generated by an opponent in a strategic situation.
This strategic risk premium has been dubbed \textit{betrayal aversion} \citep{Bohnet2004}.
Many papers find that betrayal aversion is an important determinant of trust CITE.

A recent paper has shown theoretically that the elicitation procedure of minimum acceptable probabilities (MAPs, from which betrayal aversion is identified) used in most papers leaves the door open to potential confounds such as ``ambiguity attitudes, complexity, different beliefs, and dynamic optimization'' \citep{Li2020a} if players are not rational expected utility maximizers.
Moreover, a couple of empirical papers which use more stringent identification procedures for betrayal aversion by controlling for beliefs do not find betrayal aversion \citep{Fetchenhauer2012,Polipciuc2021}, or find it to play a role for trusting only when beliefs are far more optimistic than is generally the case \citep{Engelmann2021}.

In this note, we use an online experiment to measure how much of what has been called betrayal aversion is due to distributional dependence, regardless of the source of risk being random or strategic.
To do this, we show participants complete distributions over probabilities of the good (bad) outcome of a lottery, and ask them for their cutoff value probability of the favorable outcome for preferring the lottery to a safe payoff.
Should the numerical example in \cite{Li2020a} be applicable to our setting, we expect participants to more readily accept the lottery when it comes from a distribution with a higher expected value.
We find the opposite to be true: the higher the expected value of the probability of the favorable outcome, the higher the minimum acceptable probability required by participants to accept the lottery.

While this is at odds with our expectations, it ties in with some results from the empirical literature on distributional dependence of willingness to pay (WTP).
To make this clear, we first explain how betrayal aversion is related to WTP.
\textcolor{red}{I need something here...}
Betrayal aversion is identified as the difference between two minimum acceptable probabilities: the minimum required probability of trustworthiness in order to trust in a trust game, and the minimum required probability of a favorable outcome in an equiprobable control game, where risk is generated by a randomization device.
The MAPs are elicited through a variant of the Becker-Degroot-Marschak (BDM) \citep{Becker1964} mechanism, which an often used mechanism for eliciting valuations.
There a couple of differences: (1) the auctioned good is a lottery, (2) instead of giving a maximum price for which they prefer the good to a safe payment, participants are asked to state a minimum probability of the favorable outcome of the lottery for which they prefer the lottery to a safe payment and (3) the underlying distribution of the probability of the favorable outcome is not explicitly uniform, as in the case of the distribution of potential prices for the good.

Theoretical literature has pointed out that the BDM mechanism is not incentive compatible if players are not rational expected utility maximizers \citep{Karni1987,Horowitz2006}.
This is because individuals face uncertainty regarding the price of the good and additional uncertainty about whether they will buy the good or not.
If their utility function is influenced by these uncertainties, changing the price distribution of the good might influence their MAP.

Several empirical papers find this to be the case: generally, the higher the expected price of the good, the higher the WTP \citep[for a short review of this literature, see]{Tymula2016}.
>>>>>>> beee5d6c339ef9ecc1cfdd81b98e7cfd4da32418


Imagine a lottery and a sure payment.
You know how much you receive if you win or lose the lottery, but you don't know the chances of each outcome.
Now imagine you have to write a contract specifying how good the lottery minimally has to be for you to prefer the lottery over the sure payment.
Would the requirement you set in the contract be independent of how likely you think more (less) favorable lotteries are?
As an example: would you set the same requirement in two situations, one in which favorable lotteries are more likely, and the other in which unfavorable lotteries are more likely?

From a theoretical point of view, things are clear.
If you are a rational expected utility maximizer, the requirement should be the same in both situations.
If you're not and the context influences your decisions, it is not so clear which way things will go.
Will you require a better or a lesser chance to take the lottery in a world of better opportunities?

This note reports the findings of an online experiment designed to test whether a more favorable underlying distribution of the chances of the lottery influences the threshold people require to take the lottery rather than the sure payment.
Our interest in this question was sparked by previous work on betrayal aversion.
Betrayal aversion has been identified as one of the factors influencing the decision to trust.
It is defined as an anticipatory disutility from expecting one's trust to be betrayed.
Betrayal aversion has been identified as the premium required to trust someone relative to accepting an equiprobable lottery with equal payoff consequences for an uninvolved other.

Most older papers on the subject of betrayal aversion find a positive strategic premium.
Some papers however do not.
\cite{Li2020a} propose that the original BZ design might have miss-classified the premium as betrayal aversion.
They argue that, should participants not be rational expected utility maximizers, the premium could be attributed to ``ambiguity attitudes, complexity, different beliefs, and dynamic optimization''.

In this note, we examine the effect of one of these potential confounds: different beliefs.
We do this by manipulating the underlying distributions of the lottery's winning chances.

To derive our hypotheses, we make the same assumptions as in numerical example by \cite{Li2020a} in Appendix A.\footnote{
\cite{Li2020a} use this example to show that ambiguity aversion alone may cause the strategic risk premium attributed to betrayal aversion.
}
Under these assumptions, we hypothesize that the threshold for preferring the lottery over the sure payment will be lowest in The Good treatment, followed by The Uniform treatment, and followed by The Bad treatment.\footnote{
For details, see \textcolor{red}{Appendix...}.}

Our results however clearly show the reverse ordering than predicted: participants set the lowest threshold in The Bad treatment, followed by The Uniform, followed by The Good.
This pattern has been found in a related strand of literature.
This strand examines how valuations obtained using the Becker-DeGroot-Marschak mechanism \citep{Becker1964} are influenced by the underlying price distribution.
In the Becker-DeGroot-Marschak (BDM) mechanism, a potential buyer states the maximum price for which she is willing to buy a good.
A price is drawn, and if it is lower than or equal to the price she stated, she buys the good.
If the price is higher, she keeps her endowment and does not buy the good.
\cite{Karni1987} and \cite{Horowitz2006} have shown theoretically that the mechanism is incentive compatible only if participants are rational expected utility maximizers.
Experimental studies generally find that people are willing to pay more for the same good if its price comes from a left-skewed distribution (with more mass on high prices) than when it comes from a right-skewed one CITE.

Both betrayal aversion and the design used in this note use versions of the BDM mechanism.
For this reason, the above-mentioned literature could provide an explanation for the pattern observed in our data.


\section{Procedures}\label{sec:proced}
We use three distributions, which are ordered in terms of the expected payoff over the entire distribution, as their name suggests: the Good, the Bad, and the Uniform.
Two of the three distributions were selected to emulate treatments in papers on betrayal aversion.
The first of these two (The Uniform) has equal chances of occurrence for each of the possible lotteries.
We assume that this is what participants expect to face in treatments with computer drawn lotteries, unless specified otherwise.
The second (The Bad) has an overall chance of a high payoff equal to the percentage of trustworthy respondents in papers on betrayal aversion.
The distribution in the Good treatment mirrors the one in the Bad treatment: it has the same variance, and minus the skewness of the Bad distribution.

\textcolor{red}{include table describing distributions}

To make the task easier to understand, we represent lotteries via wheels of fortune.
In each treatment, participants see the entire distribution of lotteries in that treatment, sorted in ascending order by the probability of the favorable outcome.
Figure \textcolor{red}{XZY} below shows the distribution for The Good treatment.
Dark blue sectors symbolize the high payoff, light blue sectors---the low payoff.
Participants have to state a \textit{minimum acceptable probability} (MAP): the lowest chance of a favorable outcome of the lottery such that they prefer the lottery to a sure payoff.
Since there is evidence that participants have an easier time expressing choice using integers than probabilities \citep{Quercia2016}, we ask them to answer a question which requires an integer as an input.
Specifically, they have to answer: ``Which wheels would you like to spin for your bonus?'' by inserting an integer between 0 and 15 in the blank space in the sentence: ``I prefer to spin wheels which have at least \textit{[blank]} dark blue sectors.''

The experiment was conducted online using Qualtrics.
Participants were UK residents registered on a platform for conducting academic studies (Prolific).
Since the elicitation of MAPs is rather complex \citep{Quercia2016,Polipciuc2020}, we opted for participants who had at least a bachelor's degree.
The study was pre-registered at the AEA RCT Registry (https://doi.org/10.1257/rct.7776-1.1).
275 of the 450 participants answered the eliminatory comprehension questions correctly and completed the experiment.
Those who completed the experiment (did not complete the experiment) spent a median time of 12.4 (5.9) minutes and earned 5 (1) UK pounds.




Several papers CITE find that when dealing with complex risks, participants in experiments require an extra premium compared to simple risk aversion.
This premium is positively correlated with ambiguity aversion. (ARE EFFECT size similar?)


\clearpage
\pagebreak
\bibliographystyle{apalike}
\bibliography{Communities}

\end{document}