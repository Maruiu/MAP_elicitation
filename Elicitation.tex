\begin{filecontents}{preliminary.sty}
\ProvidesPackage{preliminary}
%\DeclareOption{draft}{%
  \AtBeginDocument{%
    \renewcommand\maketitlehookc
\ProcessOptions
\RequirePackage{titling}
\endinput
\end{filecontents}

\documentclass[12pt, a4paper]{article}
\usepackage{setspace}
\usepackage{ragged2e}
\usepackage[centertags,reqno]{amsmath}
\usepackage{amssymb}
\usepackage{graphics,subfigure}
\usepackage[dvips]{graphicx}
\usepackage[hidelinks]{hyperref}
\usepackage{appendix}
\usepackage{natbib}
\usepackage{verbatim,color}
\usepackage{pdflscape}
\usepackage[showframe=false]{geometry}
\usepackage{changepage}
\usepackage{xcolor}
\usepackage{eurosym}
\usepackage{textcomp}
\usepackage[open,openlevel=1]{bookmark}
\usepackage{multirow}
\usepackage{caption}
\usepackage{hyphenat}
\newcommand{\mybox}[2]{{\color{#1}\fbox{\normalcolor#2}}}
\doublespacing

% Exception to hyphenation
\hyphenation{par-ti-ci-pants}
\hyphenation{par-ti-ci-pant}
\hyphenation{Hy-po-the-sis}
\hyphenation{ex-pe-ri-ment}
\hyphenation{ex-pe-ri-ments}

% Allows bigger tables to be scaled down
\usepackage{adjustbox}

%From my paper with Raymond
\usepackage{tabularx,calc}
\usepackage{dcolumn}                    % Aligns tables on the decimal point
\newcolumntype{d}[1]{D{.}{.}{#1}}       %       Aligns on dot
\newcolumntype{.}{D{.}{.}{3.5}}         %       Somehow it works better
\newcolumntype{C}{@{\extracolsep{.6cm}}c@{\extracolsep{0pt}}}
\usepackage{threeparttable}
\usepackage{siunitx,booktabs}
\sisetup{
    detect-all,
    round-integer-to-decimal = true,
    group-digits             = true,
    group-minimum-digits     = 4,
    group-separator          = {\,},
    table-align-text-pre     = false,
    table-align-text-post    = false,
    input-signs              = + -,
    input-symbols            = {*} {**} {***},
    input-open-uncertainty   = ,
    input-close-uncertainty  = ,
    retain-explicit-plus
}

% Commands to name appendices Appendix A, Appendix B, etc.
\makeatletter
%% The "\@seccntformat" command is an auxiliary command
%% (see pp. 26f. of 'The LaTeX Companion,' 2nd. ed.)
\def\@seccntformat#1{\@ifundefined{#1@cntformat}%
   {\csname the#1\endcsname\quad}  % default
   {\csname #1@cntformat\endcsname}% enable individual control
}
\let\oldappendix\appendix %% save current definition of \appendix
\renewcommand\appendix{%
    \oldappendix
    \newcommand{\section@cntformat}{\appendixname~\thesection\quad}
}

%Adds text specifying it is a preliminary version
\usepackage{preliminary}

\title{Eliciting Minimum Acceptable Probabilities}
\author{Martin Strobel  \and Maria Polipciuc\thanks{Maastricht University. Email: \url{m.polipciuc@maastrichtuniversity.nl}. We thank Elias Tsakas for valuable comments.}}
\date{\today	\vspace{1cm}}
\titlepage


\begin{document}
\begin{titlepage}
\clearpage\maketitle
\thispagestyle{empty}

\end{titlepage}


\section{Introduction}\label{sec:intro}


    Trial Title: Eliciting Minimum Acceptable Probabilities
    
    Country (At least one): UK/US? Representative sample: min 300 participants, extra fee, takes 2--4 days (https://researcher-help.prolific.co/hc/en-gb/articles/360019236753-Representative-Samples-on-Prolific) or only individuals with higher education? (given complexity of task?)
    
    Status
    
    Keyword (At least one)
    
    Abstract
    
This study (i) tests an underlying assumption of the Minimum Acceptable Probability (MAP) elicitation procedure introduced by \cite{Bohnet2004} and (ii) checks whether the strategy method yields similar results to the direct elicitation method in an individual decision, which represents a pared-down version of the setting in \cite{Bohnet2004}.
This MAP elicitation procedure is similar to a Becker-DeGroot-Marschak (BDM) procedure.
In BDM, a price drawn from a uniform distribution is compared to a cutoff price stated by the participant to determine the participant's payoff.
In the MAP procedure, a probability is drawn to the same effect from an unspecified distribution, about which participants may have varying subjective beliefs.
In \cite{Bohnet2004}, this distribution is generated by the actions of a pool of potential human opponents in a strategic interaction.
In this study, for simplicity, participants make an individual decision, in which they face a randomly generated ``artificial'' distribution.


The assumption which we test is necessary for the MAP elicitation to be incentive-compatible: subjects' answers are not sensitive to the distribution from which they believe the payoff-relevant probability is drawn.
However, even within expected utility, if participants have higher order risk aversion (for instance they are prudent, so they are averse to increases in downside risk---that is, they are averse to changes in a prospect's skewness) this might not be the case.


We use a between-subject design and vary treatments along two dimensions: (i) the distribution from which the payoff-relevant probability is drawn. A probability distribution function will be shown to the participants (\textcolor{red}{here we may want to also include treatments with ambiguity about this distribution}), and (ii) whether we ask participants for their MAP (which uses the strategy method, and which is a cutoff value for being willing to take a gamble rather than accept a sure payoff) or whether we ask them whether, given the distribution of chances in the gamble, they are willing to take the gamble (direct elicitation method, DM).
    
    Trial Start Date
    
    Intervention Start Date
    
    Intervention End Date
    
    Trial End Date
    
    Outcomes (End Points)
    
    Experimental Design (Public)
    
    Hypothesis.  Participants like right-skew and dislike left-skew regardless of the elicitation method (MAP or DM). For treatments eliciting MAP, this means they will have a lower MAP when he payoff-relevant probability is drawn from a distribution with a right-skew than from a distribution with a left-skew, ceteris paribus.
    
    Was the treatment clustered?
    
    Planned Number of Clusters
    
    Planned Number of Observations
    
    Was IRB approval obtained (only for ``In Development" and ``On-going" trials)? If so, also
        IRB Name
        IRB Approval Date
        IRB Approval Number






\clearpage
\pagebreak
\bibliographystyle{apalike}
\bibliography{Communities}



\end{document}