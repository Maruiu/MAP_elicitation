\documentclass[12pt, a4paper]{article}
\usepackage{setspace}
\usepackage[centertags,reqno]{amsmath}
\usepackage{amssymb}
\usepackage{graphics,subfigure}
\usepackage[dvips]{graphicx}
\usepackage[dvipsnames]{xcolor}
\usepackage[hidelinks]{hyperref}
\usepackage{appendix}
\usepackage{natbib}
\usepackage{pdflscape}
\usepackage[showframe=false]{geometry}
\usepackage{mdframed}
\usepackage{eurosym}
%\usepackage{textcomp}
\usepackage{multirow}
\usepackage{caption}
\usepackage{hyphenat}
\usepackage{tikz}
\newcommand{\mybox}[2]{{\color{#1}\fbox{\normalcolor#2}}}

\usepackage{tabularx,calc}
\usepackage{dcolumn}                    % Aligns tables on the decimal point
\newcolumntype{d}[1]{D{.}{.}{#1}}       %       Aligns on dot
\newcolumntype{.}{D{.}{.}{3.5}}         %       Somehow it works better
\newcolumntype{C}{@{\extracolsep{.6cm}}c@{\extracolsep{0pt}}}
\usepackage{threeparttable}
\usepackage{siunitx,booktabs}
\sisetup{
    detect-all,
    round-integer-to-decimal = true,
    group-digits             = true,
    group-minimum-digits     = 4,
    group-separator          = {\,},
    table-align-text-pre     = false,
    table-align-text-post    = false,
    input-signs              = + -,
    input-symbols            = {*},
    input-open-uncertainty   = ,
    input-close-uncertainty  = ,
    retain-explicit-plus
}

\begin{document}

In this paper, we use an online experiment to study whether the distribution of risks that participants face influences their willingness to take risks.
Participants can earn either a high or a low payoff by accepting a lottery, or a safe intermediary payoff by rejecting the lottery.
We do not list possible lotteries in (increasing / decreasing) order of the probability of the high payoff (similar to a Holt and Laury task risk elicitation task) and ask participants for each lottery whether they prefer it over the safe payoff.
Instead, we use a Becker--Degroot--Marschak mechanism and ask them for their minimum acceptable probability (MAP) of the lottery's high payoff for preferring the lottery over the safe payoff.

When making their decision, participants see a graphical representation of a distribution over lotteries with two possible outcomes (high and low), but varying probabilities for each outcome.
After they report their MAP, a lottery is drawn at random from the distribution.
This means in some treatments it is more likely that a lottery with a high chance of a high payoff is drawn.
We use three distributions over lotteries.
The distributions are ordered in terms of the expected payoff over the entire distribution, as their name suggests: the Good, the Bad, and the Uniform (the Good $>$ the Uniform $>$ the Bad).

We present lotteries via 32 wheels of fortune with 15 sectors each.
Dark blue sectors symbolize the high payoff (\pounds4), light blue sectors---the low payoff (\pounds1).
The sure payoff (the payoff participants receive if no wheel is spun) is \pounds2.
In each treatment, participants see the wheels sorted in ascending order by the probability of the favorable outcome, with the 32 wheels equally distributed over 4 rows.

\begin{figure}[h!]
  \centering
 {\includegraphics[width=\linewidth]{Fig1_Left_15.pdf}}
  \caption{The Good distribution}
  \label{fig:TheGood}
\end{figure}

Figure \ref{fig:TheGood} shows the distribution of lotteries for the Good treatment.
The number in each wheel is the number of sectors yielding a high payoff (ranging from 0 to 15).
The Uniform distribution has equal chances of occurrence for each of the possible wheels.
The Bad distribution has an overall chance of a high payoff equal to 0.2895.
The distribution in Good mirrors the one in Bad: its overall expected chance of a high payoff is one minus that in Bad (0.7105), it has the same variance and minus the skewness of the Bad distribution.
Table \ref{tab:distr} presents the distributions.


\begin{table}[htbp]
\centering \caption{The treatments: the distribution of chances of a high payoff}\label{tab:distr}
\begin{threeparttable}
\begin{tabular}
   {@{}
	*5c
	@{}
	}
\toprule
	&	\multicolumn{3}{c}{\# of wheels}&\\
	\cmidrule{2-4}
\# of high payoff sectors 	&	{The Good}&{The Bad}&	{The Uniform}& {Probability of high payoff}\\
\midrule
0	    &	1&       8&	2& 0   \\
1	    &	1&       4&	2& 0.07\\
2	    &	1&       4&	2& 0.13\\
3	    &	1&       3&	2& 0.20\\
4	    &	1&       2&	2& 0.27\\
5	    &	1&       1&	2& 0.33\\
6	    &	1&       1&	2& 0.38\\
7	    &	1&       1&	2& 0.47\\
8	    &	1&       1&	2& 0.53\\
9	    &	1&       1&	2& 0.60\\
10	    & 1&       1&	2& 0.67\\
11	    & 2&       1&	2& 0.73\\
12	    &	3&       1&	2& 0.80\\
13	    &	4&       1&	2& 0.87\\
14	    &	4&       1&	2& 0.93\\
15	    &	8&       1&	2& 1   \\
\midrule
Total \# of wheels	&	32&       32&	32&\\
\bottomrule

\end{tabular}
\end{threeparttable}
\end{table}

Participants are told that one of the wheels will be drawn at random, with all wheels having an equal chance to be drawn.
They are asked to state a \textit{minimum acceptable frequency}: the lowest number of dark blue sectors in the randomly drawn wheel such that they prefer to spin the wheel instead of receiving the sure payoff.\footnote{
We decided to use frequencies instead of probabilities because there is evidence that participants have an easier time expressing choice this way \citep{Quercia2016}.}
Specifically, they have to answer: ``Which wheels would you like to spin for your bonus?'' by inserting an integer between 0 and 15 in the blank space: ``I prefer to spin wheels which have at least \rule{1cm}{0.15mm} dark blue sectors.''\footnote{
We chose the setup with wheels of fortune as we wanted to make the task easy to understand.
Despite our approach being discrete, we will interpret the frequencies ($x$ out of 15) as minimum acceptable probabilities.
}

We use a within-subject design.
One of the three decisions participants make (the three MAPs) is relevant for payment.

\clearpage
\pagebreak
\bibliographystyle{apalike}
\bibliography{MAPs}
\end{document}
